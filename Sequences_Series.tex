\documentclass[12pt]{article}
\usepackage[utf8]{inputenc}
\usepackage{geometry}
\geometry{a4paper, top=1.25cm,bottom=1.5cm,right=1.5cm,left=1.5cm}
\usepackage{amsmath, amssymb, amstext}
\usepackage{tikz}
\setlength\parindent{0pt}
\tikzset{myBound/.style={color=blue}}
\tikzset{myAxisLine/.style={line width=0.3mm,color=black}}
\tikzset{myLine/.style={thick,color=black}}
\usepackage{enumitem}
\usepackage{array}
\newcolumntype{M}{> {$} l < {$}}
\usepackage{mathtools}
\newcommand\Mycomb[2]{\prescript{#1\mkern-1mu}{}C_{#2 \mkern+6mu}}
\renewcommand{\baselinestretch}{1.2}
\usepackage{tkz-euclide}
\usetikzlibrary{positioning}

\usepackage{hyperref}
\hypersetup{
	colorlinks=true,
	linkcolor=blue,
	filecolor=magenta,      
	urlcolor=cyan,
}

\definecolor{minGridCol}{RGB}{130, 255, 255}
\tikzset{myMajGrid/.style={color=gridCol, line width=0.0mm}}
\tikzset{myMinGrid/.style={color=minGridCol, line width=0mm}}
\tikzset{myAxisLine/.style={line width=0.3mm,color=black}}

\usepackage{tkz-euclide}

\tikzset{myAngle/.style={thin,color=black}}
\newcommand\onelabel[6]{
	\tkzMarkAngle[myAngle,size =#1]({#2},{#3},{#4})
	\tkzLabelAngle[pos=#5 ]({#2},{#3},{#4}){\footnotesize ${#6}$}
}

\title{Sequences and Series}
\author{Kh notes}
\date{}



%----------------------Triangles------------------------------------------
\newcommand{\x}{0}
\newcommand{\y}{0}
\newcommand\triangleSAS[7]{
	\renewcommand{\x}{0};
	\renewcommand{\y}{0};
	\coordinate (R) at (#5,#6);
	%\filldraw[red](\x,\y) circle [radius=1pt];
	%\filldraw[](0,0) circle[radius=1pt];
	\begin{scope}[ xshift=#5cm, yshift=#6cm,rotate around={#4:(R)}, scale around={#7:(R)} ]
		%\filldraw[blue](\x,\y) circle [radius=1pt];
		\coordinate(O) at(0,0);
		\coordinate(B) at (#3,0);
		\coordinate(A) at ({#1*cos(#2)},{#1*sin(#2)});		
	\end{scope}
	
}


%----------------------Bounding Boxes------------------------------------------
\newcommand\brect[2]{
	\draw[myBound](-#1,-#2)rectangle(#1,#2);
	%backcolour
	\draw[myBound, opacity=0.1, xstep=0.5cm, ystep=0.5cm](-#1,-#2)grid(#1,#2);
	\draw[myBound](0,0)circle[radius=2pt];
}
\begin{document}
	\maketitle
\tableofcontents
\newpage
\section{Introduction (Number Sequences)}
\section{Arithmetic Sequences}
\subsection{Ex 5B.1}
\subsection{Ex 5B.2}
\section{Geometric  Sequences}
\subsection{Ex 5C}
\section{Growth and Decay}
Starter Questions:
\begin{enumerate}
	\item A school had 1200 students and a year later this has increased by 8\% . How many students are now in the school?
	\item Mary buys a car for \$40,000 and in one year its price has decreased by 12\%. What is the value of it now?
	\item The population of Sydney is currently 5.2 million. If it increases at a rate of 1.25\% annually, what will the population be after 3 years?
\end{enumerate}
\subsection{Ex 5D}
\section{Financial Mathematics}
\subsection{Compound Interest}
\hrule\vspace{0.5cm}

\LARGE $$u_n = u_0(1+i)^n$$ \normalsize

\hspace{1cm}$u_0$ Initial Investment (Principal)

\hspace{1cm}$i$ Interest rate per compounding period

\hspace{1cm}$n$ Number of periods

\hspace{1cm}$u_n$ The final value of the investment

\vspace{0.5cm}\hrule
\subsubsection{Ex 5E.1}
\subsection{Inflation}
\subsubsection{Ex 5E.2}
\subsection{Real Value of an Investment}
\subsubsection{Ex 5E.3}
\subsection{Depreciation}
\subsubsection{Ex 5E.4}
\subsection{Using Financial Models}
\subsubsection{Ex 5E.5}
\section{Series}
\subsection{Sigma Notation}
\subsubsection{Ex 5F}
\subsection{Arithmetic Series}
\subsubsection{Ex 5G}
\subsection{Finite Geometric Series}
\subsubsection{Ex 5H}
\subsection{Infinite Geometric Series}
\subsubsection{Ex 5I}
\end{document}