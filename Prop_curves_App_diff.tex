\documentclass[10pt]{report}
\usepackage[utf8]{inputenc}
\usepackage{geometry}
\geometry{a4paper, top=1.25cm,bottom=1.5cm,right=1.5cm,left=1.5cm}
\usepackage{amsmath, amssymb, amstext}
\usepackage{tikz}
\setlength\parindent{0pt}
\tikzset{myBound/.style={color=blue}}
\tikzset{myAxisLine/.style={line width=0.3mm,color=black}}
\tikzset{myLine/.style={thick,color=black}}
\usepackage{enumitem}
\usepackage{array}
\newcolumntype{M}{> {$} l < {$}}
\usepackage{mathtools}
\newcommand\Mycomb[2]{\prescript{#1\mkern-1mu}{}C_{#2 \mkern+6mu}}
\renewcommand{\baselinestretch}{1.2}
\usepackage{tkz-euclide}
\usetikzlibrary{positioning}

\definecolor{minGridCol}{RGB}{130, 255, 255}
\tikzset{myMajGrid/.style={color=gridCol, line width=0.0mm}}
\tikzset{myMinGrid/.style={color=minGridCol, line width=0mm}}
\tikzset{myAxisLine/.style={line width=0.3mm,color=black}}

\usepackage{tkz-euclide}

\tikzset{myAngle/.style={thin,color=black}}
\newcommand\onelabel[6]{
	\tkzMarkAngle[myAngle,size =#1]({#2},{#3},{#4})
	\tkzLabelAngle[pos=#5 ]({#2},{#3},{#4}){\footnotesize ${#6}$}
}

\title{Properties of curves and applications of differentiation }
\author{Kh notes}
\date{}



%----------------------Triangles------------------------------------------
\newcommand{\x}{0}
\newcommand{\y}{0}
\newcommand\triangleSAS[7]{
	\renewcommand{\x}{0};
	\renewcommand{\y}{0};
	\coordinate (R) at (#5,#6);
	%\filldraw[red](\x,\y) circle [radius=1pt];
	%\filldraw[](0,0) circle[radius=1pt];
	\begin{scope}[ xshift=#5cm, yshift=#6cm,rotate around={#4:(R)}, scale around={#7:(R)} ]
		%\filldraw[blue](\x,\y) circle [radius=1pt];
		\coordinate(O) at(0,0);
		\coordinate(B) at (#3,0);
		\coordinate(A) at ({#1*cos(#2)},{#1*sin(#2)});		
	\end{scope}
	
}


%----------------------Bounding Boxes------------------------------------------
\newcommand\brect[2]{
	\draw[myBound](-#1,-#2)rectangle(#1,#2);
	%backcolour
	\draw[myBound, opacity=0.1, xstep=0.5cm, ystep=0.5cm](-#1,-#2)grid(#1,#2);
	\draw[myBound](0,0)circle[radius=2pt];
}
\begin{document}
	\maketitle
	\tableofcontents
	\newpage
	\chapter{Properties of curves}
\section{Review Questions}
\subsection{Some fundamental derivatives:}
\def\arraystretch{2}
\setlength\tabcolsep{1cm}
\newcommand{\sep}{0.25cm}
\begin{center}
	\begin{tabular}{| M | M |} \hline
		\text{Function} & \text{Derivative} \\\hline
		f(x)=x^n  & \displaystyle f'(x)=nx^{n-1}~~ (n \in \mathbb{R})\\ [\sep] \hline
		f(x)=e^x  & \displaystyle f'(x) =e^x\\ [\sep] \hline
		f(x)=\ln x  & \displaystyle f'(x) =\frac{1}{x}\\ [\sep] \hline
		f(x)=\sqrt{x}  & \displaystyle f'(x) =\frac{1}{2 \sqrt{x} }\\ [\sep] \hline
		f(x)=\sin{x}  & \displaystyle f'(x) =\cos{x}\\ [\sep] \hline
		f(x)=\cos{x}  & \displaystyle f'(x) =-\sin{x}\\ [\sep] \hline
		f(x)=\tan{x}  & \displaystyle f'(x) =\sec^2{x}\\ [\sep] \hline
	\end{tabular}
\end{center}
\subsection{Rules of differentiation:}
Chain Rule:
\begin{align*}
	y&= g(u_{(x)})\\
	\frac{dy}{dx} &= g'(u_{(x)} )u'_{(x)}
\end{align*}
Product Rule:
\begin{align*}
	y&= u_{(x)}v_{(x)}\\
	\frac{dy}{dx} &= u_{(x)}v'_{(x)} +u'_{(x)}v_{(x)}
\end{align*}
Quotient Rule:
\begin{align*}
	y&= \frac{u_{(x)}}{v_{(x)}}\\
	\frac{dy}{dx} &= \frac{ u'_{(x)}v_{(x)} - u_{(x)}v'_{(x)} }{ [ v_{(x)} ]^2 }
\end{align*}



\section{Start Q and A}
\begin{tikzpicture}[xscale= 0.6,yscale=0.6]
	%\draw[myMinGrid, xstep=1cm, ystep=1cm](-14,-5) grid(14,6);
	\draw[myAxisLine, <->](-14,0)--(14,0)node[right]{$x$};
	\draw[myAxisLine, <->](0,-5)--(0,7)node[above]{$y$};
	\draw[black](0,0) circle [radius=5pt];
%	\foreach \n [count=\i] in{-12,-10,...,-2,2,4,..., 12}{
%		\draw[](\n,-0.1)node[below]{\footnotesize \n};
%		\draw[](\n,-0.1)--(\n,0.1);}
%	\foreach \n [count=\i] in{-4,-2,2,4,...,5}{
%		\draw[](-0.1,\n )node[left]{\footnotesize \n};
%		\draw[](-0.1,\n )--(0.1,\n);}
	\draw[yscale=1,xscale=1,domain=-11:12,smooth,variable=\x,blue, thick] plot ({\x},{  0.01*(\x+9)*(\x-9)*(\x-1)    });
	\draw[](9,3)node[left]{$y=f(x)$};
	%	\draw[yscale=1,xscale=1,domain=5:13,smooth,variable=\x,blue, thick] plot ({\x},{  (\x-9)^2    });
	%\draw[yscale=1,xscale=1,domain=-3:5,smooth,variable=\x,blue, thick] plot ({\x},{  (\x-1)^2    });
\end{tikzpicture}
\section{Tangents}
The tangent to a curve at a point A is the best approximating straight line to the curve at point A.\\

(Leibniz definition) Tangent to the curve $y=f(x)$ at the point $(a,f(a))$ is the line through the infinitely close pair of points either side of $f(a)$\\

$$\frac{y-f(a)}{x-a} =\lim\limits_{h\to 0}\frac{f(a+h)-f(a)}{h}$$\\


It is a single point of contact with the curve (although it may intersect the curve at some other point)\\


For the function $y=f(x)$, and some $x=a$\\
$(a,f(a))$ is on the curve \\
$f'(a)$ is the gradient of the curve at $x=a$\\\\
$\displaystyle \frac{y-f(a)}{x-a}=f'(a)$\\\\
$\Rightarrow y= f'(a)(x-a) + f(a)$ is the equation of the tangent line\vspace{2cm}

\textbf{4 worked examples} 
\newpage
\section{Normals}
\newcommand{\ang}{50}
\newcommand{\ext}{5}
The product of the gradients of perpedicular lines = -1
\begin{center}
\begin{tikzpicture}[xscale= 0.6,yscale=0.6]
	%\draw[myMinGrid, xstep=1cm, ystep=1cm](-14,-5) grid(14,6);
	\draw[myAxisLine, <->](-5,0)--(5,0)node[right]{$x$};
	\draw[myAxisLine, <->](0,-5)--(0,5)node[above]{$y$};
	%\draw[black](0,0) circle [radius=5pt];
	\draw[yscale=1,xscale=1,domain=-\ext:\ext,smooth,variable=\t,blue, thick] plot (  {\t*cos(\ang)},{ \t*sin(\ang)    });
	\draw[yscale=1,xscale=1,domain=-\ext:\ext,smooth,variable=\t,red, thick] plot (  {\t*cos(\ang + 90)},{ \t*sin(\ang + 90)    });
	\coordinate (O) at (0,0);
	\coordinate (A) at (  {\ext*cos(\ang)} ,{ \ext*sin(\ang)    });
	\coordinate (X) at ( {\ext*cos(\ang)} ,0);
	\coordinate (A') at (  {\ext*cos(\ang + 90)},{ \ext*sin(\ang +90)    });
	\coordinate (X') at ({\ext*cos(\ang + 90)},0);
	\onelabel{0.6}{X}{O}{A}{0.85}{\theta}
	
	\draw[dashed, blue](A) -- (X)node[midway, right]{\scriptsize $\sin(\theta)$};
	\draw[dashed, blue](O) -- (X)node[midway, above, xshift=0.3cm]{\scriptsize $\cos(\theta)$};
		\tkzMarkRightAngle[myLine,size=.4](A',O,A)
			\draw[dashed, red](A') -- (X')node[midway, left]{\scriptsize $\sin( \theta +\frac{ \pi }{2} )$};
		\draw[dashed, red](O) -- (X')node[midway, above, xshift=-0.1cm]{\scriptsize $\cos(\theta + \frac{ \pi }{2} )$};
		
		\draw[blue](8,4)node[right]{\Large $m=\frac{\sin(\theta)}{\cos(\theta)}$};
		\draw[red](8,2)node[right]{\Large $m_{\perp}=\frac{\sin(\theta +\frac{ \pi }{2} )}{\cos(\theta +\frac{ \pi }{2})}= - \frac{\cos(\theta)}{\sin(\theta)}$};
			\draw[black](8,0)node[right]{\Large $m \times m_{\perp} = -1$};
\end{tikzpicture}\vspace{1cm}

\begin{tikzpicture}[xscale= 0.6,yscale=0.6]
	%\draw[myMinGrid, xstep=1cm, ystep=1cm](-14,-5) grid(14,6);
	\draw[myAxisLine, <->](-14,0)--(14,0)node[right]{$x$};
	\draw[myAxisLine, <->](0,-5)--(0,7)node[above]{$y$};
	\draw[black](0,0) circle [radius=5pt];
	\draw[yscale=1,xscale=1,domain=-11:12,smooth,variable=\x,blue, thick] plot ({\x},{  0.01*(\x+9)*(\x-9)*(\x-1)    });
	\draw[yscale=1,xscale=1,domain=-11:12,smooth,variable=\x,green, thick, dashed] plot ({\x},{  0.01*(3*(\x)^2 -2*\x -81)    });
	\draw[yscale=1,xscale=1,domain=3:11,smooth,variable=\x,red, thick] plot ({\x},{  0.52*\x -5.56   });
	\draw[yscale=1,xscale=1,domain=4:10,smooth,variable=\x,orange, thick] plot ({\x},{ (-1/0.52)*\x + 11.542  });
	\draw[](9,3)node[left]{$y=f(x)$};
	
		\draw[black](7,-1.92) circle [radius=5pt];

\end{tikzpicture}


\end{center}
\section{Increasing and Decreasing}
\section{Stationary Points}
\subsection{Turning points (minima, maxima)}
\subsection{Stationary points of inflection}
\section{Shape}
\section{Inflection Points}
\section{Understanding functions and their derivatives}

\chapter{Applications of differentiation}







\newpage
% I prefer the alignment at the equal sign.
\begin{alignat*}{2}
	&\text{The equation is:}\qquad & 9a-4 &= 14+3a\\
	&\text{Subtract $3a$:} & 6a-4 &= 14\\
	&\text{Subtract 4:} & 6a &= 18\\
	&\text{Divide by 6:} & a &= 3
\end{alignat*}

$$A\widehat{B}C$$
$$A\widehat{BC}C$$
$$A\hat{B}C$$
$$N \tilde{a}$$
$$X \sim \mathcal{N}(\mu,\,\sigma^{2})$$








\end{document}